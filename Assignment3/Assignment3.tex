%%%%%%%%%%%%%%%%%%%%%%%%%%%%%%%%%%%%%%%%%%%%%%%%%%%%%%%%%%%%%%%
%
% Welcome to Overleaf --- just edit your LaTeX on the left,
% and we'll compile it for you on the right. If you open the
% 'Share' menu, you can invite other users to edit at the same
% time. See www.overleaf.com/learn for more info. Enjoy!
%
%%%%%%%%%%%%%%%%%%%%%%%%%%%%%%%%%%%%%%%%%%%%%%%%%%%%%%%%%%%%%%%


% Inbuilt themes in beamer
\documentclass{beamer}

% Theme choice:
\usetheme{CambridgeUS}

\setbeamertemplate{caption}[numbered]{}

\usepackage{enumitem}
\usepackage{tfrupee}
\usepackage{amsmath}
\usepackage{amssymb}
\usepackage{gensymb}
\usepackage{graphicx}
\usepackage{txfonts}

\def\inputGnumericTable{}

\usepackage[latin1]{inputenc}                                 
\usepackage{color}                                            
\usepackage{array}                                            
\usepackage{longtable}                                        
\usepackage{calc}                                             
\usepackage{multirow}                                         
\usepackage{hhline}                                           
\usepackage{ifthen}
\usepackage{caption} 
\captionsetup[table]{skip=3pt}  
\providecommand{\pr}[1]{\ensuremath{\Pr\left(#1\right)}}
\providecommand{\cbrak}[1]{\ensuremath{\left\{#1\right\}}}
\renewcommand{\thefigure}{\arabic{table}}
\renewcommand{\thetable}{\arabic{table}}

% Title page details: 
\title{AI1110: Assignment 3} 
\author{SRI CHARVI\\BT21BTECH11008}
\date{\today}
%\logo{\large \LaTeX{}}

\providecommand{\pr}[1]{\ensuremath{\Pr\left(#1\right)}}

\begin{document}

% Title page frame
\begin{frame}
    \titlepage 
\end{frame}

% Remove logo from the next slides
%\logo{}


% Outline frame
\begin{frame}{Outline}
    \tableofcontents
\end{frame}


% Lists frame
\section{Question}
\begin{frame}{Question}

\textbf{a)} In the coin experiment, the space $S$ consists of the outcomes $h$ and $t$:

\begin{center}
$S=\{h,t\}$
\end{center}


  and it's events are the four sets \{\emptyset\}, \{t\}, \{h\}, S. 
  
  $P\{h\}=p$ and $P\{t\}=q$. Show that $p+q=1$\\
  
  \textbf{b)}We consider now the experiment of the toss of a coin three times. What is the probability of getting two heads in the first two tosses?
\end{frame}
\section{Solution (a)}
\begin{frame}{Solution (a)}
	The given coin is fair.
	
	We get either heads or tails for one toss.
	
	Number of possible outcomes = 2 
	
	Probability of getting heads is $P\{h\}=p$
	
	Probability of getting tails is $P\{t\}=q$
	
	Both events are exhaustive, mutually exclusive and equally probable.
    \begin{align}
            &\implies  P\{h\}=  P\{t\}= \dfrac{1}{2}\\
            &\implies p=q=\dfrac{1}{2}
    \end{align}
    
    \begin{align}
        \therefore p+q &= \dfrac{1}{2}+\dfrac{1}{2}\\
            &=1
    \end{align}
\end{frame}

\section{Solution (b)}
\begin{frame}{Solution (b)}
    Let P(E) be the required probability
    
	The possible outcomes/events of this experiment(tossing the coin thrice) are:
	\begin{center}
	    $hhh$, $hht$, $hth$, $htt$, $thh$, $tht$, $tth$, $ttt$
	\end{center}
	
	
	In this case, the probability of each elementary event equals \dfrac{1}{8}.
	
	Thus the probability P $\{hhh\}$ that we get three heads equals \dfrac{1}{8}. The event
	
	\begin{equation}
	   \{E\} = \{hhh, hht\}
	\end{equation}
	
	consists of the two outcomes $hhh$ and $hht$.
	
	Hence the required probability is
	\begin{align}
	    P(E) &=  P \{hhh\} + P \{hht\}\\
	        &= \dfrac{1}{8}+\dfrac{1}{8}\\
	        &=2/8
	\end{align}
\end{frame}	
	
	

\end{document}